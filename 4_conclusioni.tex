\section{Conclusioni}
\label{section:conclusioni}

In conclusione i risultati posti come obiettivo sono stati raggiunti.

Innanzitutto è stato studiato quale modello di grafo fosse più in linea con il tipo di 
analisi che questo progetto tratta.
La topologia di grafo proposta da Dorogovtsev e Mendes ha avuto risultati di condivisione migliori. 
Il parametro chiave è definito dalla sua struttura particolare che 
prevede un numero di connessioni quasi doppio rispetto al Preferential Attachment.

Il secondo studio invece ha fornito una risposta concreta a come una informazione con un certo argomento 
viene condivisa in un Social Network con differenti utenti.
L'argomento viene ``deciso'' tramite 5 valori che vanno ad incidere sulla probabilità di condivisione della notizia.
Si può affermare che, più la notizia ha importanza, più questa viene condivisa.
I test effettuati non sono stati esaustivi, ma sono bastati per capire la tendenza appena affermata.
Effettuare tutti i test sarebbe stato poco produttivo e avrebbe impiegato un'enormità di risorse.
Si vuol far notare infatti che per affrontare tutte le possibilità, dati i 5 parametri, ci sarebbero volute
\[
  \left(\frac{1 - 0}{0.05}\right)^5 * 500 test = 1.600.000.000
\]
simulazioni complete per garantire la stessa precisione. 

Per quanto concerne il terzo studio, sono stati confrontati due gruppi di utenti con differenti 
proprietà di astensione alla notizia. Queste proprietà sono state descritte dalla funzione di probabilità 
Weibull che, grazie ad un parametro $\alpha$, permette di modificare la densità della distribuzione.
Così facendo si avrà un gruppo con una probabilità di condivisione maggiore dell'altro.
Nel test effettuato è stato mostrato un nuovo risultato. 
La forza della notizia in questo caso non influisce con la crescita dei ``Viewers''.
Il parametro importante è l'aumentare degli individui con la bassa astensione alla notizia rispetto a quelli
con una più alta.
Si pone anche l'attenzione come, in alcuni casi, pochi utenti che condividono molto possono avere un'influenza 
maggiore di tanti che condividono meno.


\subsection{Sviluppi Futuri}

Come già affermato nell'introduzione, questo lavoro è stato già affrontato in passato e continuerà ad
esserlo dato l'importanza che, l'argomento discusso, ha nella società moderna.



%Come abbiamo potuto osservare sono pienamente in linea con 
% motivare il singolo grafo



Prendere i risultati del 3\degree test e confrontarli con quelli dei test base e vedere come si comportano, minore? maggiore?




















\section{Conclusioni}
\label{section:conclusioni}

In conclusione i risultati posti come obiettivo sono stati raggiunti.

Innanzitutto è stato studiato quale modello di grafo fosse più in linea con il tipo di 
analisi che questo progetto tratta.
La topologia di grafo proposta da Dorogovtsev e Mendes ha avuto risultati di condivisione migliori. 
Il parametro chiave è definito dalla sua struttura particolare che 
prevede un numero di connessioni quasi doppio rispetto al Preferential Attachment.

Il secondo studio invece ha fornito una risposta concreta a come una informazione con un certo argomento 
viene condivisa in un Social Network con differenti utenti.
L'argomento viene ``deciso'' tramite 5 valori che vanno ad incidere sulla probabilità di condivisione della notizia.
Si può affermare che, più la notizia ha importanza, più questa viene condivisa.
I test effettuati non sono stati esaustivi, ma sono bastati per capire la tendenza appena affermata.
Effettuare tutti i test sarebbe stato poco produttivo e avrebbe impiegato un'enormità di risorse.
Si vuol far notare infatti che per affrontare tutte le possibilità, dati i 5 parametri, ci sarebbero volute
\[
  \left(\frac{1 - 0}{0.05}\right)^5 * 500 test = 1.600.000.000
\]
simulazioni complete per garantire la stessa precisione. 

Per quanto concerne il terzo studio, sono stati confrontati due gruppi di utenti con differenti 
proprietà di astensione alla notizia. Queste proprietà sono state descritte dalla funzione di probabilità 
Weibull che, grazie ad un parametro $\alpha$, permette di modificare la densità della distribuzione.
Così facendo si avrà un gruppo con una probabilità di condivisione maggiore dell'altro.
Nel test effettuato è stato mostrato un nuovo risultato. 
La forza della notizia in questo caso non influisce con la crescita dei ``Viewers''.
Il parametro importante è l'aumentare degli individui con la bassa astensione alla notizia rispetto a quelli
con una più alta.
Si pone anche l'attenzione come, in alcuni casi, pochi utenti che condividono molto possono avere un'influenza 
maggiore di tanti che condividono meno.


\subsection{Sviluppi Futuri}

Come già affermato nell'introduzione,considerata la rilevanza dell'argomento nella società moderna,
questo tema, nonostante sia già stato affrontato in passato, continuerà ad esserlo anche in futuro.

Vi sono ancora molti test da effettuare ed alcuni temi su cui si potrebbe lavorare sono di seguito elencati.
\begin{itemize}
 \item Un primo studio potrebbe puntare ad un'estensione dove si inserisca la durata della notizia, intesa come visibilità. 
 Nella maggior parte dei Social Network esiste una ``home'' dove vengono visualizzate le notizie 
 condivise dagli altri utenti a cui si è connessi.
 Queste notizie, però, risultano avere una durata massima; essa viene definita in base alla data di condivisione della stessa e
 dal grado di interazione degli altri utenti, ovvero più una notizia viene condivisa più questa diventa importante e rimane in vista.
 
 \item Un secondo studio potrebbe essere quello di rielaborare i test da un punto di vista differente. 
 In questo lavoro si assume che la notizia venga condivisa inizialmente dal nodo con il maggior numero di vicini, quindi, 
 paragonandolo alla realtà si potrebbe definire l'hub come un personaggio noto ad un pubblico vasto.
 Nell'espansione che si sta definendo il paziente zero potrebbe essere estratto casualmente tra tutti i nodi, o mediante una decisione più complessa.
 
 \item Un terzo studio potrebbe riguardare la rete sociale: in questo lavoro, a causa della divisione dei due applicativi, 
 le simulazioni fanno riferimento ad uno stesso grafo di 500 nodi, grafo creato grazie all'algoritmo proposto da Dorogovtsev e Mendes. 
 Considerato quanto appena esplicato, potrebbe risultare opportuno analizzare nuovamente gli obiettivi con più reti, studiandone perciò le differenze.
 
 \item Un ultimo studio potrebbe puntare a confrontare i risultati del 2\degree e 3\degree test effettuati in questo lavoro;  
 si potrebbe quindi analizzare le differenti crescite, di modo da osservarne il comportamento in più situazioni. 
\end{itemize}
















